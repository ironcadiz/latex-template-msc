This chapter consists of two sections, the first one shows an overview some of the
different methods that have been previously used in the literature
for understanding or estimating urban perception, these methods are separated
into 3 types: the classic approaches (all the methods not relying on
massive amounts of data are grouped here), approaches based on machine learning and
approaches consisting of machine learning models combined with other techniques.
The different methods are explained and a brief analysis of advantages and disadvantages
is provided for each of them. Section two summarizes the main aspects of the research
on explainability on deep learning, and describes some techniques that have been applied
in urban perception or other domains that are relevant for this work.


\section{Solutions for estimating urban perception.}

\subsection{Classic approaches.}
Methods for measuring perception of urban spaces have appeared in the literature of several
disciplines for many years,  with some of the most influential studies dating back to 1960
\cite{lynch}. Due to technological limits the literature consisted mainly of several types of
qualitative surveys for a long time. This surveys consisted in having subjects, complete
different tasks such as drawing maps of a certain place \cite{lynch}, evaluating fundamental
aspects of a neighborhood \cite{nasar_perception}, or in more recent approaches evaluating
the impact of transformations generated with edited images \cite{jiang_minimizing}. Most of
these surveys were conducted in person or by phone, and then the results were analyzed by hand.







\subsection{Pure machine learning approaches.}

\subsection{Mixed approaches.}

\section{ Explainability in deep learning.}