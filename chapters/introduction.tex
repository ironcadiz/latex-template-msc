\section{Thesis outline.}
This work consists of seven chapters, first is this introduction, which
shows the  importance of both the automatic calculation of urban perception, and
explainability in artificial intelligence, and ends with the hypothesis
of this research. Chapter 2 is a detailed description of the research objectives.
Chapter 3 consists of a summary of the relevant previous research. In chapter 4
the problem is formally defined and the proposed model is described. Chapter 5 gives
details on model implementation and training. Finally, in chapter 6 and 7 consist of the results and
conclusions of the research.

\section{Importance of automatic urban perception}

\textit{Urban perception} is a feeling held by people about a location. These feelings can be and
are often related to a particular characteristic, like happiness or beauty, or also
inherently negative ones, like insecurity or fear \cite{tamara_judgments}.

The visual urban perception is  responsible for a large parte of the experience that people
go through while  being at or using an urban space, this not only affects how much the spaces
themselves are used \cite{khisty} but also the use of related means of transport \cite{antonakos}.
Other studies have also found correlations between urban perception, crime statistics \cite{tamara_judgments}
and wealth, and therefore used it as a proxy measure of inequality \cite{tamara_judgments,hidalgo_inequality, rossetti}.

On the other hand, being able to understand a community's need and perception of a city at scale is something
of key importance on developing cities, in order for the limited  resources of local governments to be applied
appropriately \cite{santani}, but traditional methods for the measuring of urban perception, consist of hand made polls
about specific locations making it a extremely costly and hard to escalate process \cite{clifton}.

Considering this facts, automatic estimation of urban perception at great scales becomes a very relevant
research problem, because the generated data would be a powerful tool to guide the improvement of
public spaces and the design of public policy.

\section{Importance of explainability in artificial intelligence}

\section{Hypothesis}