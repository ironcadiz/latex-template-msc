% Select one class
%\documentclass{pucthesis}		% For DVI
\documentclass[pdftex]{pucthesis}	% For pdfLaTeX
%\documentclass[spanish]{pucthesis}		% For DVI, in spanish
%\documentclass[pdftex,spanish]{pucthesis}	% For pdfLaTeX, in spanish

%%%%%%%%%%%%%%%%%%%%%%%%%%
%      Nota: si usas español, algunos nombres      %
%      debes cambiarlos manualmente en este     %
%        documento. En teoría, nunca deberías       %
%           modificar el archivo pucthesis.cls              %
%%%%%%%%%%%%%%%%%%%%%%%%%%

%%%%%%%%%
%   Packages	 %
%%%%%%%%%
\usepackage{ textcomp }
% Floats
\usepackage{graphicx}
\usepackage{float}
\floatstyle{boxed}
\restylefloat{figure}
\usepackage{subfigure}
\usepackage{color}

% Math packages
\usepackage{amsmath}
\usepackage{amsfonts}
\usepackage{amssymb}

% Closest font to Times New Roman
\usepackage{times}

% To make pretty tables
\usepackage{booktabs}
\usepackage{multirow}

% To avoid underfull errors in the bibliography
\usepackage{etoolbox}
\apptocmd{\sloppy}{\hbadness 10000\relax}{}{}

% To make cites and references
\usepackage[hidelinks,pdfusetitle,pdfdisplaydoctitle]{hyperref}
\usepackage[notocbib]{apacite} 
\usepackage{doi}
\renewcommand{\doitext}{}

%--------- NEW ENVIRONMENTS --------- You are free to remove or use it
\newtheorem{definition}{\bf Definition}[chapter]
\newtheorem{property}{Property}[chapter]
\newtheorem{claim}{Claim}[chapter]
\newtheorem{lemma}{\bf Lemma}[chapter]
\newtheorem{proposition}{Proposition}[chapter]
\newtheorem{theorem}{\noindent \bf Theorem}[chapter]
\newtheorem{corollary}{\bf Corollary}[chapter]
\newtheorem{pf}{Proof}[chapter]
\newtheorem{example}{\bf Example}[chapter]
\newtheorem{remark}{Remark}[chapter]

\makeatletter
  \setlength{\beftitle}{105\p@\@plus24\p@}
  \setlength{\afttitle}{65\p@}
\makeatother

\begin{document}

\mdate{July 30, 2020}         % date manuscript changed
\version{1}                                     % manuscript version #

\title[Deep Neural Network Models with explainable components for urban space perception.]
   {\bf Deep Neural Network Models with explainable components for urban space perception.}       
\author[Andrés Cádiz Vidal]{Andrés Cádiz Vidal}

\address{Escuela de Ingenier\'ia\\
                   Pontificia Universidad Cat\'olica de Chile\\ 
                   Vicu\~na Mackenna 4860\\
                  Santiago, Chile\\
                  {\it Tel.\/} : 56 (2) 354-2000}
\email{username@uc.cl}

\facultyto    {the School of Engineering}
%\department   {}
\faculty                             {Faculty of Engineering}
\degree                  {Master of Science in Engineering}  %   {Mag\'ister en Ciencias de la Ingenier\'ia}  %        
                                 % or {Doctor in Engineering Sciences}    % {Doctor en Ciencias de la Ingenier\'ia}
\advisor                            {Hans Löbel}
\committeememberA {Patricio de la Cuadra}
\guestmemberA            {Member B}
\ogrsmember                 {Member C}
\subject                            {Structural Engineering}
\date                                 {July 2020}
\copyrightname             {Andrés Cádiz Vidal}
\copyrightyear               {MMXX}
\dedication                      {Gratefully to my parents and siblings}

%%%%%%%%%%%%%%%%%%%%
%       PRELIMINARIES                              %
%----------------------------------------------------%
%      page i & ii: cover page                   %
%      page iii: dedication                         %
%%%%%%%%%%%%%%%%%%%%

\NoChapterPageNumber
\pagenumbering{roman}
\maketitle

%%%%%%%%%%%%%%%%%%%%
%   EXTRA PAGES                                     %
%----------------------------------------------------%
%      page --: not used                             %
%%%%%%%%%%%%%%%%%%%%

%\newpage
%\thispagestyle{empty}

%----------------------------------------------------------------------%

%%%%%%%%%%%%%%%%%%%%%%%%%
%      page v: ACKNOWLEDGEMENTS                   %
%%%%%%%%%%%%%%%%%%%%%%%%%

\phantomsection \label{acknowledgements} % Comment if hyperref is unused
\chapter*{ACKNOWLEDGEMENTS}           
Thanks to my family and girlfriend for their unmeasurable support during the time
this research was done. Thanks to my advisor for his great guidance and contributions.
Thanks to all the members of \textit{IALAB PUC} and \textit{Instituto Milenio Fundamentos de los Datos}
project \#4 for all the very useful feedback and conversations. Thanks to ANID (former CONICYT) for
granting me the National Masters scholarship, which enabled me to complete my studies without
any issues.





\cleardoublepage

%----------------------------------------------------------------------%

%%%%%%%%%%%%%%%%%%
%          page v & up ---                      %
%            Table of contents              %
%            List of figures                     %
%            List of tables                      %
%%%%%%%%%%%%%%%%%%

\pdfbookmark{\contentsname}{toc}
\tableofcontents
\phantomsection \label{listoffigures}
\listoffigures
\phantomsection \label{listoftables}
\listoftables
\cleardoublepage

%----------------------------------------------------------------------%

%%%%%%%%%%%%%%%%%%%%%%%%
%      page x & xi: ABSTRACT - RESUMEN        %
%%%%%%%%%%%%%%%%%%%%%%%%

\phantomsection \label{abstract}
\chapter*{ABSTRACT}
Urban perception has been an important research subject for at least 60 years, with studies
being conducted by many different disciplines, using a variety of methodologies mainly based
on surveys over either real or simulated urban environments. The recently increased availability
of large amounts of data and highly scalable data collection methods powered by the modern web
has allowed for new techniques from other domains to be extended to the estimation of urban perception.
In particular, machine learning methods used as either stand alone models or feature extraction tools
have proven to be very effective for automatic quantification of the perception. This methods (neural networks in particular) present the disadvantage of having a black box
nature, which can make it hard for humans to understand the obtained results, therefore limiting
their application.

In this work we present a novel neural network architecture for automatic urban perception quantification.
Our best model, named AttnSegRank, can output an estimated urban perception score based on an image,
along with a set of weights (displayable as a heatmap) that reflect the importance of each part of the image
on the calculation of the score. It achieves this by including the output of a pretrained
semantic segmentator leveraged with an attention mechanism as part of the architecture. The model
we show in this work presents very similar performance with those in the previous literature but
with a much better interpretability, making it  not only a more useful model for urban perception
measuring and research, but a contribution to explainability in
the deep learning and computer vision fields that can be applied to other tasks as well.


\vfill
\noindent {\bf Keywords}:  urban perception, deep learning, explainable artificial intelligence.

\cleardoublepage

\phantomsection \label{resumen}
\chapter*{RESUMEN}
La percepción urbana ha sido un tema de investigación importante por al menos 60 años, con trabajos
hechos desde una variedad de disciplinas, usando distintas metodologías principalmente basadas en encuestas
sobre ambientes urbanos reales o simulados. Recientemente, el surgimiento de las tecnologías web ha aumentado
masivamente la disponibilidad de grandes volúmenes de datos y de técnicas de recolección de datos altamente escalables
permitiendo que técnicas pertenecientes a otros dominios sean utilizadas para estimar la percepción urbana. En
particular, algunos métodos de aprendizaje de máquina, usados ya sea como modelos completos o como herramientas
de extracción de características, han demostrado ser muy efectivos para la cuantificación automática de la percepción.
Estos métodos (en particular las redes neuronales) presentan la desventaja de tener una naturaleza caja negra,
que dificulta la capacidad de entender los resultados obtenidos desde el punto de vista humano, lo que
limita su aplicabilidad.

En este trabajo presentamos una nueva arquitectura de red neuronal para la cuantificación de la percepción urbana.
A partir de una imagen nuestro mejor modelo, \textit{AttnSegRank}, entrega como resultado una estimación de la percepción,
junto a un conjunto de pesos que  (visualizables como un mapa de calor) que reflejan la importancia de cada parte de
la imagen en el calculo del score. Esto se lo logra incluyendo un segmentador semántico combinado con un mecanismo
de atención como parte de la arquitectura de la red. El modelo que mostramos en este trabajo logra un rendimiento similar
a los existentes en la literatura pero con  mucho mejor interpretabilidad,  haciendolo no solo un modelo más útil para
medición e investigación de la percepción urbana, si no que una contribución a la explicabilidad en aprendizaje profundo
y visión por computador que se puede aplicar a otras tareas.

\vfill
\noindent {\bf Palabras Claves}: percepción urbana, aprendizaje profundo, inteligencia artificial explicable.

\cleardoublepage

%%%%%%%%%%%%%
%   TEXT  OF THESIS     %
%%%%%%%%%%%%%

\pagenumbering{arabic}

\chapter[INTRODUCTION]{Introduction}
\textit{Urban perception} is a feeling held by people about a location. These feelings can be and
are often related to a particular characteristic, like happiness or beauty, or also
inherently negative ones, like insecurity or fear \cite{tamara_judgments}. Understanding the
cause of these feelings is a complex task, since unique social and psychological aspects
of each individual affect how they perceive and the spaces they observe \cite{nasar_perception}.

Visual urban perception is  responsible for a large parte of the experience that people
go through while  being at or using an urban space, this not only affects how much the spaces
themselves are used \cite{khisty} but also the use of related means of transport \cite{antonakos}.
Other studies have also found correlations between urban perception, crime statistics \cite{tamara_judgments}
and wealth, and therefore used it as a proxy measure of inequality \cite{tamara_judgments,hidalgo_inequality, rossetti}.

On the other hand, being able to understand a community's need and perception of a city at scale is something
of key importance on developing cities, so that the limited  resources of local governments can be applied
more efficiently \cite{santani}.

Traditional methods for obtaining this type of data, consist of hand made polls
about specific locations making systematic evaluation of perception  an extremely costly and hard
to escalate task \cite{nasar_perception,clifton}. Other approach consist of surveys based on
computer generated images of simulated spaces, this is more scalable, but is limited to experimental
design and it doesn't apply to a real space \cite{lain_grenspace,iglesias_perception}.

Currently, thanks to the great volumes of data generated by web platforms \cite{hidalgo_inequality}
and to modern deep learning (DL) and computer vision techniques \cite{lecun_dl}, new solutions for
estimating urban perception have become feasible, and some previous studies have achieved
significant results, either by applying traditional deep learning \cite{hidalgo_placepulse}
or by combining it with other approaches \cite{rossetti, zhang_measuring}. The solutions consist
mainly of training deep convolutional neural network models (DCNN) \cite{lecun_mnist} with datasets
of urban images that have some sort of label that is used as an estimator for the perception
of that urban space. Most of the research is based on the place pulse dataset \cite{hidalgo_placepulse}, which
consists of pairs of images along with labels that indicate which of the images is more representative of a
particular attribute.

However, current deep learning methodologies, have the disadvantage of being "black boxes", in other
words, they lack a direct or systematic way to explain or interpret the obtained results. This problem
comes from the end to end nature of the neural network models and from the millions of learnable parameters
they contain. Many of the problems in which these models are used would greatly benefit of more
human understandable explanations of the results, making this a very important area of
research for the deep learning field \cite{adadi_xai}.

For the particular case of urban perception, explainability of the results is highly relevant, since
the added information is valuable for the design of public policy, for example, it could be use to
better discriminate which locations would be better recipients of an intervention, and which elements
to modify so it convenes an effective improvement of perception.

Current research in explainability is primarily moving in two directions: one is to design novel neural network architectures and training methods so
the models are more interpretable, such as the work by \citeA{yinpeng_semantic}, the other
direction is to  create post-hoc algorithms \cite{adadi_xai} that analyze the results given by the
neural network, these algorithms sometimes use other machine learning models, including neural networks,
such as the work by \citeA{ghorbani_xai}.

The work by \citeA{rossetti}, presents an approach to this problem for the urban perception case
by using semantic segmentation of the images \cite{segnet} as input for a discrete choice model that
estimates the perception. The approach allows for a post-hoc aggregated analysis of the results, since
the weights of the model are measure of the importance of each class of the semantic segmentation in the
calculation of the perception.

The objective of this work is to design and train a model for the urban perception problem,
that can give explainable insights on an instance level. For that it proposes a novel solution,
consisting of a neural network architecture, that is end-to-end trainable and by using semantic
segmentation \cite{pspnet} and self attention mechanisms \cite{vaswani_attention} can show
explainable insights for each of the input images.

ESTO HAY QUE ARREGLARLO AL FINAL \\
The remainder of this manuscript is organized as follows, Chapter 3 summarizes relevant previous research. In chapter 4
the problem is formally defined and the proposed model is described. Chapter 5 gives
details on model implementation and training. Finally, in chapter 6 presents the research results
and 7 the final conclusion.


\chapter[RELATED WORK]{Related Work} \label{work}
This chapter consists of two sections, the first one shows an overview some of the
different methods that have been previously used in the literature
for understanding or estimating urban perception, these methods are separated
into 3 types: the classic approaches (all the methods not relying on
massive amounts of data are grouped here), approaches based on machine learning and
approaches consisting of machine learning models combined with other techniques.
The different methods are explained briefly and a short discussion is presented.
Section two summarizes the main aspects of the research
on explainability on deep learning, and describes some techniques that have been applied
in urban perception or other domains that are relevant for this work.


\section{Solutions for estimating urban perception.}

\subsection{Classic approaches.}
Methods for measuring perception of urban spaces have appeared in the literature of several
disciplines for many years,  with some of the most influential studies dating back to 1960
\cite{lynch}. Due to technological limits the literature consisted mainly of several types of
qualitative surveys for a long time. This surveys consisted in having subjects, complete
different tasks such as drawing maps of a certain place \cite{lynch}, evaluating fundamental
aspects of a neighborhood \cite{nasar_perception}, or in more recent approaches evaluating
the impact of transformations generated with edited images \cite{jiang_minimizing}. Most of
these surveys were conducted in person or by phone, and then the results were analyzed manually,
making it very difficult and costly to scale to multiple locations, or larger amounts of samples.
The main benefit of this approach, is that it permits a very refined control of the observation process
since both the subjects being interviewed and the spaces in question are chosen by the researcher.
Added to that, the experiments conducted in person allow for the observer to use senses different
than vision to analyze the subject space, resulting in a richer appreciation.

Other methodology, more common in economics and engineering, consists of using discrete choice models
and stated choice surveys to model the effect of different variables in perception or other urban
related variables  \cite{rose_sc, iglesias_perception, torres_housing}. The amount and complexity of the
variables measured depends on the model design. To have and exact control of the variables that
have an effect on the survey, computer generated images of urban spaces can be used
\cite{iglesias_perception,torres_housing}.

The advantage of this method is that through the estimated parameters of the model, the effect
of each of the studied variables on the perception estimation can be measured, allowing for
quantitative results and an understanding of the impact different elements have on the
perception of the urban landscape. The main disadvantage of this approach comes from the
difficulty of the  survey design, variables need to be chosen carefully and the process its
vulnerable to biases from the model designer.

\subsection{Pure machine learning approaches.}

Thanks to the massive adoption of web and mobile technologies such as google maps, new types of
data are available in considerably large volumes, and new highly scalable ways of  generating data can be
designed and implemented quickly. That fact allows for some very data dependent machine learning
algorithms to be applied to new  problems, including urban perception estimation. Several different
datasets have been proposed for this problem, most of them based on surveys over large amounts of urban images
\cite{hidalgo_inequality, hidalgo_placepulse, quercia_aesthetic, liu_machine, santani}. The most important
of them, all consisting of pairwise comparisons of street view images, are \textit{Place pulse 1.0} (PP 1) \cite{hidalgo_inequality}
with measures of safety, class and uniqueness over images of 4 cities, \textit{Urban Gems} with measures of
beauty, quietness and happiness over images of London and \textit{Place pulse 2.0} (PP 2) \cite{hidalgo_placepulse}, the largest dataset
available, with measures of six different attributes over images of 56 different cities, the models proposed on this work are
trained on this dataset. All of these were collected through public online surveys of large scale, where the users
are asked to choose the image most representative of an attribute of a pair, see figure \ref{fig:survey} for an example.

\begin{figure}[ht]
	\begin{center}
	\includegraphics[width=0.5\textwidth]{./figures/placepulse.png}
	\caption[Place pulse 2.0 survey]{Snapshot of the place pulse 2.0 survey. Extracted from \citeA{hidalgo_placepulse} }
	\label{fig:survey}
	\end{center}
\end{figure}

Earlier attempts at using this data for training models tried to turn the problem into a classification problem
by  ranking the images from the votes with manually engineered methods such as the one suggested on the
place pulse 1.0 paper \cite{hidalgo_inequality} and use the rank to split the data in two halves with a different
label, \citeA{tamara_judgments} use this approach to train SVM models on PP 1 using different types of visual features as input,
including a deep neural network. On the PP 2 paper, the authors present the first end to end deep learning model for
urban perception regression, which uses a typical transfer learning technique \cite{survey_transfer}, a
Imagenet \cite{imagenet} pretrained  network for the base of the model, which is used as input
for by two parallel modules, one for classification and one for regression. They train the architecture
separately on the 6 different attributes of the dataset, the models learn to emulate human voting and
to output a urban perception score (through the regression module) on the image for the correspondent attribute.
Other works \cite{porzi_predicting, santani} take similar approaches but pretrain models or use features based on
the places dataset \cite{zhou_places}, which provides better performance according to their results.


\citeA{zhang_measuring}, train models on PP 2 by combining a DCNN features and a SVM classifier, they use this model to
obtain perception indicators of Beijing, they also use a semantic segmentation model \cite{cordts_cityscapes} on the images and used the results
as input to a linear regression, interpreting the regression weights as an indication of importance of the different segmentation
classes on perception. On a following work \cite{zhang_uncovering} they train one deep network to predict all 6 attributes of PP 2
in one forward pass, they do this using an end-to-end architecture similar to \citeA{hidalgo_placepulse} but adding
one output and loss component for each attribute.

Is important to note that most of the literature so far is more focused on applying the models to new cities
\cite{zhang_measuring, santani, costa_lisbon, rossetti} or generating new datasets with new attributes
\cite{santani, zhang_uncovering}, than it is on improving model  design and performance.
This is consistent with the fact that so far no good measures of performance for this problem have been defined,
due to the fact that the datasets don't provide a measure of perception per se but a proxy through the survey votes.
The objective of the models in the literature is to rank the images by the estimated perception of an attribute, but
they measure  performance on the task of emulating the human votes, which doesn't necessarily correlate with the
models capacity to generalize and rank well, especially in conflicted cases where even human voters
would have difficulties \cite{zhang_measuring}.

\subsection{Mixed approaches.}



\section{ Explainability in deep learning.}

\chapter[DATASET]{Dataset} \label{dataset}
This chapter presents an analysis of the dataset used. On Section \ref{sec:dataset_desc} an overall description
and main statistics are shown. Section \ref{sec:dataset_prep} presents an early analysis of the data along with
the preprocessing steps taken. Finally section 3 makes an analytical definition of the problem
of learning urban perception from this data.


\section{Description}
\label{sec:dataset_desc}
As was mentioned previously, this work is based on the Place Pulse 2.0 dataset \cite{hidalgo_placepulse}.
PP 2.0 is a crowdsourced dataset designed for learning urban perception from street view like images.
Unlike regular datasets for supervised machine learning, that have labels for each image, Place Pulse
consists of pairwise comparisons between images, and the ground truth is a vote representing which of
the images is more representative of an attribute (ties are also possible). That structure makes
traditional classification / regression approaches inapplicable, but opens the door for
pairwise based ranking techniques, that are more suitable to urban perception since a ground truth for
how much an image represents an abstract attribute such as "safety" it's impossible to define.

The dataset consists of approximately 1.2 million pairwise comparisons of 112,000 images from 56 cities,
distributed on 6 attributes: wealthy, safety, depressing, boring, lively and beautiful, making it
the biggest available dataset for urban perception. The crowdsourcing survey was
active for 5 years and it was answered by 81,630 different users. Demographic information
about the users was not collected.

\section{Analysis and preprocessing}
\label{sec:dataset_prep}

As a first preprocessing step all noisy images are removed by using a file size threshold,
since files small in size are mostly  google api errors or unintelligible places like
dark tunnels.

It is important to note that, unlike most crowdsourced datasets, the authors of PP did not
perform a validation on the votes. 99.59\% of the image pairs that appear in the
data set have a single vote in a category (see \ref{fig:rep_hist} for details.),
making it impossible to corroborate if they are reasonable by comparing the votes of multiple people.
Even though previous research indicates that answers to this surveys aren't affected by user
bias or demographics \cite{hidalgo_inequality, costa_lisbon}, the inconsistency in the votes is a
clear dataset disadvantage: 34\% of the  pairs that have more than one vote in an attribute
show inconsistencies between the votes.


\begin{figure}[ht]
	\begin{center}
	\includegraphics[width=0.8\textwidth]{./figures/rep_hist.png}
	\caption[Repetition histogram]{ Histogram for amount of repetitions for each pair of images }
	\label{fig:rep_hist}
	\end{center}
\end{figure}

For this work we completely remove all inconsistent duplicates and keep a single instance of those consistent.
After these, steps 1,207,938 votes for 111,299 images are left. See Table \ref{tab:votes}
for the exact vote distribution.

\begin{table}[H]
	\begin{center}
	\caption[Votes Distribution]{ Vote distribution after preprocessing.}
	\begin{tabular}{ll}
		\hline
		\textbf{Attribute} & \textbf{\# of votes} \\ \hline
		Wealthy            & 150,370               \\
		Safety             & 364,130               \\
		Depressing         & 130,781               \\
		Boring             & 125,744               \\
		Lively             & 263,123               \\
		Beautiful          & 173,790               \\ \hline
		Total              & 1,207,938            \\
	\end{tabular}
	\label{tab:votes}
	\end{center}
\end{table}

Users of the survey had the possibility of voting that a pair is tied for an attribute,
meaning that they didn't perceive any significant difference. Previous works usually
discard this data and don't use it for learning, focusing only on the votes where a preference was
chosen \cite{hidalgo_placepulse,zhang_measuring,tamara_judgments}. After preprocessing 15.3\% of the
votes are ties, which means a significant amount of information is lost by disregarding them.
Due to that we decided to add additional rules to the learning problem in order to be able to use
these votes for learning. Details are shown on the following chapter.

\chapter[PROPOSED MODEL]{Proposed Model} \label{arch}
This chapter presents a detailed explanation of the neural network models proposed in this work and
the correspondent baselines used for comparison. In section one the architectures of
the main networks are shown. In section 2 the loss functions used are described. Finally,
section 3 shows the baselines models used for the ablation study of both performance and
explainability.

\section{Network architectures}
As was mentioned before, the main principle followed for model design is to enhance explainability
while maintaining performance as much as possible. With that in mind, we combine two
state of the art techniques from the deep learning literature, semantic segmentation
and attention mechanisms to design three novel architectures that present a significant
improvement in explainability over traditional blackbox CNNs. We describe these architectures
in the following sub sections, ordered by model complexity.

\subsection{Segmentation as a feature extractor.}
The traditional deep learning approach in computer vision, consists of using a pretrained
CNN \cite{lecun_mnist}, on the Imagenet dataset \cite{imagenet}, such as the ResNet \cite{he_resnet},
usually called the feature extractor, and then stacking a custom set of layers over its output features. Leaving the CNN weights
fixed or updating them on training  depends on the particular problem. This is the approach taken
by most of the previous literature on urban perception \cite{hidalgo_placepulse,tamara_judgments,zhang_measuring}.

In this work, we propose replacing the traditional feature extractors for a fully trained semantic segmentation
network. The semantic segmentation task consists of assigning a label to every pixel in an image, and therefore
it implies a fine grained detection of object edges, providing a rich amount of information that is human understandable.
The output of a semantic segmentation model is a probability distribution over the different classes for each pixel,
making it usable as a feature map of the image. See figure \ref{fig:segmentation} for an example.

We base our models on the PSPNet architecture \cite{pspnet}, since it is one of the highest performing models
available in the literature. It's design its based on a ResNet50 and a pyramid pooling module, which consists on
parallel poolings and convolutions at different scales, that are then concatenated and used to generate the output with a
final convolution and a softmax layer.

\begin{figure}[ht]
	\begin{center}
	\includegraphics[width=0.8\textwidth]{./figures/segmentation.png}
	\caption[Example of Semantic Segmentation]{Examples of semantic segmentation by the PSPNet model on the CityScapes dataset. Extracted from \citeA{pspnet} }
	\label{fig:segmentation}
	\end{center}
\end{figure}

\begin{figure}[ht]
	\begin{center}
	\includegraphics[width=0.9\textwidth]{./figures/pspnet.png}
	\caption[PspNet architecture]{PSPNet architecture. Extracted from \citeA{pspnet} }
	\label{fig:segmentation}
	\end{center}
\end{figure}


We train PspNet on the CityScapes dataset \cite{cordts_cityscapes}, since its urban images taken from a car have
considerable similarity to street view images, and its classes have proven informative for the urban perception problem
in previous research \cite{rossetti,zhang_measuring}.  After this process we keep the network weights fixed
and use the output as a features for subsequent layers.

\subsection{Self attention.}

\subsection{Segmentation as attention query.}

\section{Loss function}

\section{Baselines}

\subsection{ResNet50 + MLP}

\subsection{ResNet50 + Attention layers + MLP}

\chapter[METHODOLOGY]{Methodology} \label{methodology}
This chapter shows the practical details of the implemented models and their
respective training process.

All of our models are implemented using the Pytorch library
\cite{paszke_pytorch} version 1.2.0.
We use the implementation and pretrained weights of ResNet
available on the Torchvision library \cite{marcel_torchvision}.
We train our own PSPNet based on the implementation by \citeA{huang_psp}.
All models are trained using a single 12 Gb Nvidia Geforce-GTX 1080 Ti GPU except
for the mixed model, which is trained on a 24 Gb Nvidia Titan RTX.

For training we make a 75\%/25\% train/validation splits of the dataset for
each attribute. We keep the splits fixed for all models, so they all see
and are evaluated on the same data. All models are trained for 40 epochs
and we keep the model with the best validation accuracy on epoch end.

\begin{table}[H]
    \begin{center}
        \caption[Hyper parameters]{Hyper parameters and configurations for each model.}
        \begin{tabular}{|l|r|r|r|r|r|}
            \hline
            \textbf{Parameter/Model} & \textbf{ResNet50} & \textbf{ResnetAttn} & \textbf{Seg}    & \textbf{Seg + Attn} & \textbf{Seg Att} \\ \hline
            Batch Size     & 32                           & 32                                & 32                         & 32                                   & 32                           \\
            Learning Rate  & $10^{-4}$                    & $10^{-4}$                         & $10^{-4}$                  & $10^{-4}$                            & $10^{-4}$   \\
            Opt. Algorithm & SGD                          & SGD                               & Adam                       & Adam                                 & Adam                         \\
            Finetuning     & Yes                          & Yes                               & No                         & No                                   & Yes                          \\
            Dropout        & 0.3                          & 0.3                               & 0                          & 0                                    & 0.1                          \\
            Semantic Dropout        & N/A                          & N/A                               & 0                          & 0.1                                    & 0.1                          \\
            Weight Decay   & $10^{-5}$                    & $10^{-5}$                         & 0                          & 0                            & 0    \\
            \hline
        \end{tabular}
        \label{tab:hyperparams}
    \end{center}
\end{table}

Baselines are trained with SGD with a momentum of $0.9$ \cite{rumelhart_backprop}
as it provided better results empirically. For segmentation based models we train with
Adam \cite{kingma_adam} and we set $\epsilon$, $\beta_1$ and $\beta_2$ to $10^{-9}$, $0.9$ and $0.98$ respectively.
We use semantic dropout on both models that have segmentation and attention, and add
an equivalent regular dropout layer to the ResNetAttn Baseline for fair comparison.
Weight decay and traditional dropout are used for all baseline models that finetune ResNet weights.
See table \ref{tab:hyperparams} for details on the training hyperparameters.


\chapter[RESULTS]{Results} \label{results}
This chapter shows the main results obtained. On section one we present the quantitative
performance and training results. Section 2 explains how the different visualizations
are generated, including examples for all the models.

\section{Quantitative results}

\subsection{Model performance}

Even though the objective of this research is to learn a ranking (or regression) to
quantify the urban perception, exact labels for this are not available, so we have to measure
model performance based on the Place Pulse votes, which as was mentioned on section
\ref{sec:problem_def}, has considerable issues. We use as performance measure the
equivalent to classification accuracy, considering which image won the vote as the target label.
In other words, we evaluate the percentage of restrictions (see \ref{eq:constraints})
that are satisfied by the model. We do this separately for each attribute in it's corresponding
validation set and the final accuracy value for each model is calculated as the mean accuracy through
all attributes.

Both ResNet based baseline models achieve an accuracy of \texttildelow 66\% and
as it was expected, replacing the more expressive CNN features for semantic segmentation,
caused a significant performance drop, falling to 60.62\% for SegRank and 61.43\% %TODO: update that value
for SelfSegRank. %TODO: AttnSegRank.
See table BLABLA, for the exact accuracy values.

\subsection{Training behavior}

Models trained on the place pulse dataset are very prone to overfitting, we believe this is due to it
having a very large amount of votes in comparison to the amount of available images, and because the task
is very hard to generalize given the high amount of noise that the dataset has from how it was collected.
This can be seen clearly on figures \ref{fig:resnet_graph} and \ref{fig:resnet_attn_graph}. Both
baselines models present considerably overfitting, showing accuracy differences between seen and unseen data
of up to 25\%, and ceasing to improve on the validation set after one or two training epochs.

\begin{figure}[ht]
	\begin{center}
	\includegraphics[width=1\textwidth]{./figures/resnet50_graph.png}
	\caption[ResNet Training curves]{
        ResNet50 baseline accuracy vs epoch learning curves on training (a) and validation (b).
        }
	\label{fig:resnet_graph}
	\end{center}
\end{figure}

\begin{figure}[ht]
	\begin{center}
	\includegraphics[width=1\textwidth]{./figures/resnet_attn_graph.png}
	\caption[ResNetAttn Training curves]{
        ResNetAttn baseline accuracy vs epoch learning curves on training (a) and validation (b).
        }
	\label{fig:resnet_attn_graph}
	\end{center}
\end{figure}

Replacing the CNN features for semantic segmentation generates a considerable change in training
behavior, with the reduced expressiveness of the segmentation acting as a very strong regularizer,
overfitting completely disappears, which translates to a drop of around 20\% to 30\% accuracy in training,
but of only 6\% on validation.

The basic SegRank architecture still reaches convergence after one or two epochs.
Adding the self attention layer makes it slightly slower allowing the model reach a higher validation accuracy.


\begin{figure}[ht]
	\begin{center}
	\includegraphics[width=1\textwidth]{./figures/segrank_graph.png}
	\caption[SegRank Training curves]{
        SegRank accuracy vs epoch learning curves on training (a) and validation (b).
        }
	\label{fig:segrank_graph}
	\end{center}
\end{figure}

\begin{figure}[ht]
	\begin{center}
	\includegraphics[width=1\textwidth]{./figures/selfsegrank_graph.png}
	\caption[SelfSegRank Training curves]{
        SelfSegRank accuracy vs epoch learning curves on training (a) and validation (b).
        }
	\label{fig:selfsegrank_graph}
	\end{center}
\end{figure}

As it can be seen on the results of all models, the learning process is consistent through out
the different attributes. The accuracy of the different attributes is also consistent across the
different models, with boring and beautiful being the hardest and easiest tasks to learn respectively
on all models.


\section{Visualization results}

- Per attribute visualizations.

- Per object visualizations

- Baseline vs segrank vs segattn

\chapter[DISCUSSION]{Discussion} \label{discussion}
In this chapter we will do further analyses of the model results with particular
emphasis on the AttentionSegRank architecture due to it's better performance
and explainability. On section one we will discuss the effects of using semantic segmentation
on neural network  training. Section 2 will show the quantitative relationship between
segmentation, attention and the perception quantification. And finally, section 3 we will
analyze the implications of this method on model explainability.

\section{Effect of semantic segmentation on learning}
As was already mentioned on section \ref{sec:training}, adding a fixed segmentator to
the neural network architecture resulted in a reduction of performance along with a considerable
reduction of overfitting. The behavior was expected when fully replacing the CNN features,
due to the reduced expressiveness of the segmentation and the lack of finetuning, but
unexpectedly, although it is reduced, this behavior persists when combining the fixed
segmentation with the finetuned ResNet50 through the attention layer. We conclude from this
that restricting the attention weights to the shapes and classes given by the segmentation
has a regularizing effect on learning, reducing the model capacity even when the amount of trainable
weights is maintained.

In the case of the PlacePulse dataset this is not a problem since
all traditional deep models suffer of significant overfitting. It remains an interesting research
question  if these behaviors will transfer to other tasks and datasets.

\begin{figure}[ht]
	\begin{center}
	\includegraphics[width=1\textwidth]{./figures/wealthy_graph.png}
	\caption[Wealthy Training curves]{
        Wealthy accuracy vs epoch learning curves on training (a) and validation (b).
        }
	\label{fig:wealthy_graph}
	\end{center}
\end{figure}


\section{Relationship between urban perception and semantic segmentation}

\section{Relationship between urban perception and attention}

\section{Effect of attention over semantic segmentation on model explainability}

\chapter[CONCLUSIONS]{Conclusions} \label{conclusions}
Thanks to the massive increase on availability of large amounts of data and the advancements
of deep learning, new ways to approach old problems have become possible in a wide
range of fields. This models usually provide more effective and generalizable solutions,
but take away a large portion of the result interpretability and the capacity to understand
what the models are truly doing. The modelling of urban perception has made use
of these advancements with good success, but the subjective nature of the problem
makes it a task that is specially affected by the lack of explainability of modern
machine learning algorithms. Due to that the recent literature has presented hybrid methods
combining regression or discrete choice models with high level features extracted
with pretrained neural networks such semantic segmentation or object detection.
This techniques provide a better understanding of what is happening inside the model
but sacrifice the higher expressiveness and performance that neural networks
provide when trained end to end.

In this work we presented a novel neural network architecture, aimed at tackling this problem,
through the use of semantic segmentation combined with standard deep learning methods like
fine tuning convolutional features and attention mechanisms. This model is capable of
successfully estimating the perception  with a similar performance than those in the literature,
but at the same time it outputs the attention weights and the segmentation of the image providing
additional data that is human interpretable. We also present an aggregated analysis and visualizations
of the results, that show that attention weights are a good tool for augmenting model explainability.

\section{Contribution to the state of the art.}

\subsection{Enhancing explainability through high level features and attention.}
The main contribution of this research is an end to end trainable neural network
architecture for urban perception quantification that presents very desirable
explainability properties. Unlike previous approaches in urban perception our models
have the capacity to generate explainable insights on an instance level thanks to the
semantic attention weights, making it a considerably more powerful tool both
for research and practical application.

Additionally, the analysis on an aggregated level of segmentation and attention weights
allows to draw similar conclusions to those from previous research, that based their models
on more simple and interpretable techniques such as linear regression or econometric models,
meaning that our architecture achieves an at least as good level of explainability
but with a significantly more expressive and performing deep neural network.

\subsection{Semantic segmentation as part of a neural network.}
The AttentionSegRank neural architecture that we propose, contributes with a novel way of combining
semantic segmentation with traditional deep convolutional features through the attention operations,
allowing the network to learn the capacity of dynamically choosing which parts of an image to attend
based on the semantic classes of the segmentation. This idea is not exclusive to the urban perception task
nor to ranking problems, and may be used for classification or regression in any computer vision task
that allows for a pretrained segmentator to be used or that has available data to train a new one from
scratch.

We think that this approach could prove useful on other vision domains where an improvement
in model explainability is needed without sacrificing too much performance.

\section{Future research directions.}



%%%%%%%%%%%%%
%       REFERENCES        %
%%%%%%%%%%%%%

\cleardoublepage
\phantomsection \label{references}
\bibliographystyle{apacite}
\renewcommand{\bibname}{REFERENCES}

%%%% ACTIVAR SIGUIENTES 3 LINEAS SI POSTGRADO RECHAZA LA BIBLIOGRAFIA
%\setlength{\bibleftmargin}{0em}
%\setlength{\bibindent}{0em}
%\setlength{\bibitemsep}{1em}


\bibliography{Thesis}

%%%%%%%%%%%%
%      APPENDICES      %
%%%%%%%%%%%%

\appendix % It is like a chapter, so each appendix (A, B, C...) must to be considered as a section

\newpage
\section[Semantic Segmentation]{Semantic Segmentation}
\subsection{Visual representation}
\label{sec:seg_colors}
For visually representing segmentation, we make a color map over the images, following
the cityscapes color palette \cite{cordts_cityscapes}. See Figure \ref{fig:segmentation_colors}
for the exact palette and class list, and Figure \ref{fig:cs_sample} for an example.

\begin{figure}[ht]
	\begin{center}
	\includegraphics[width=0.2\textwidth]{./figures/seg_colors.png}
	\caption[Segmentation color palette]{
        Segmentation color palette.
        }
	\label{fig:segmentation_colors}
	\end{center}
\end{figure}

\begin{figure}[ht]
	\begin{center}
	\includegraphics[width=1\textwidth]{./figures/cityscapes_sample.png}
	\caption[CityScapes sample]{
        CityScapes sample.
        }
	\label{fig:cs_sample}
	\end{center}
\end{figure}

\subsection{Segmentation distribution}
\label{sec:seg_distribution}
Given the domain shift from Cityscapes to PlacePulse, is important to check how
the segmentation behaves on the new dataset.

The most significant difference between the datasets is the image size, which are almost 13
times bigger in cityscapes, allowing for smaller objects like traffic signs to be clearly
distinguishable. In training CS images are used with size of $769 \times 769$, while
place pulse images are used with the standard $224\times224$. Another important difference
is the origin of the images, while Cityscapes images were all taken on different cities of
the same developed country (Germany), PlacePulse images come from 56 cities distributed on all
continents, including both developed and developing countries, with the later ones contributing
images with a significant visual difference.

Table \ref{tab:segmentation} show the percentage of pixels belonging to each segmentation class
on the entire set of images of the PlacePulse dataset. Evidently this are not ground truth labels,
but the ones obtained by our  PspNet trained on Cityscapes. As was expected, classes representing
physically smaller objects have an almost negligible contribution since the smaller image size renders
them pretty much unidentifiable. Domain shift makes the model constantly confuse the sidewalk (which can
be seen in a large percentage of PlacePulse images) with the main road, reducing the class presence
to a very underwhelming 0.96\%. Same behavior can be perceived with the terrain class.


\begin{table}[H]
	\begin{tabular}{|l|r|} \hline
	Segmentation Class & \% of pixels \\ \hline
	Building      & 26.60\% \\
	Vegetation    & 25.52\% \\
	Road          & 24.21\% \\
	Sky           & 6.24\%  \\
	Fence         & 5.09\%  \\
	Truck         & 2.96\%  \\
	Car           & 1.94\%  \\
	Person        & 1.48\%  \\
	Bicycle       & 1.36\%  \\
	Motorcycle    & 1.28\%  \\
	Sidewalk      & 0.96\%  \\
	Wall          & 0.89\%  \\
	Terrain       & 0.65\%  \\
	Pole          & 0.33\%  \\
	Train         & 0.26\%  \\
	Bus           & 0.14\%  \\
	Traffic sign  & 0.05\%  \\
	Traffic light & 0.02\%  \\
	Rider         & 0.02\%  \\ \hline
	\end{tabular}
	\caption[Segmentation Distribution]{
		Pixel distribution of the segmentation classes over the PlacePulse Dataset
	}
	\label{tab:segmentation}
\end{table}

\end{document}
