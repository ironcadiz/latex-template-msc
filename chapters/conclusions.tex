Thanks to the massive increase on availability of large amounts of data and the advancements
of deep learning, new ways to approach old problems have become possible in a wide
range of fields. This models usually provide more effective and generalizable solutions,
but take away a large portion of the result interpretability and the capacity to understand
what the models is truly doing. The modelling of urban perception has taken advantage
of these advancements, with good success but the subjective nature of the problem
makes it a task that is specially affected by the lack of explainability of modern
machine learning algorithms. Due to that the recent literature has presented hybrid methods
combining regression or discrete choice models with high level features extracted
with pretrained neural networks such semantic segmentation or object detection.
This techniques provide a better understanding of what is happening inside the model
but sacrifice the higher expressiveness and performance that neural networks
provide when trained end to end.

In this work we presented a novel neural network architecture, aimed at tackling this problem,
through the use of semantic segmentation combined with standard deep learning methods like
fine tuning convolutional features and attention mechanisms. This model is capable of
successfully estimating the perception  with a similar performance than those in the literature
but at the same time it outputs the attention weights and the segmentation of the image providing
additional data that is human interpretable. We also present an aggregated analysis and visualizations
of the results, that show that attention weights are good tool for augmenting model explainability.


\section{Contribution to the state of the art}

\section{Future research directions}