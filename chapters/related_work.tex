This chapter consists of two sections, the first one shows an overview some of the
different methods that have been previously used in the literature
for understanding or estimating urban perception, these methods are separated
into 3 types: the classic approaches (all the methods not relying on
massive amounts of data are grouped here), approaches based on machine learning and
approaches consisting of machine learning models combined with other techniques.
The different methods are explained and a brief analysis of advantages and disadvantages
is provided for each of them. Section two summarizes the main aspects of the research
on explainability on deep learning, and describes some techniques that have been applied
in urban perception or other domains that are relevant for this work.


\section{Solutions for estimating urban perception.}

\subsection{Classic approaches.}
Methods for measuring perception of urban spaces have appeared in the literature of several
disciplines for many years,  with some of the most influential studies dating back to 1960
\cite{lynch}. Due to technological limits the literature consisted mainly of several types of
qualitative surveys for a long time. This surveys consisted in having subjects, complete
different tasks such as drawing maps of a certain place \cite{lynch}, evaluating fundamental
aspects of a neighborhood \cite{nasar_perception}, or in more recent approaches evaluating
the impact of transformations generated with edited images \cite{jiang_minimizing}. Most of
these surveys were conducted in person or by phone, and then the results were analyzed manually,
making it very difficult and costly to scale to multiple locations, or larger amounts of samples.
The main benefit of this approach, is that it permits a very refined control of the observation process
since both the subjects being interviewed and the spaces in question are chosen by the researcher.
Added to that, the experiments conducted in person allow for the observer to use senses different
than vision to analyze the subject space, resulting in a richer appreciation.

Other methodology, more common in economics and engineering, consists of using discrete choice models
and stated choice surveys to model the effect of different variables in perception or other urban
related variables  \cite{rose_sc, iglesias_perception, torres_housing}. The amount and complexity of the
variables measured depends on the model design. To have and exact control of the variables that
have an effect on the survey, computer generated images of urban spaces can be used
\cite{iglesias_perception,torres_housing}.

The advantage of this method is that through the estimated parameters of the model, the effect
of each of the studied variables on the perception estimation can be measured, allowing for
quantitative results and an understanding of the impact different elements have on the
perception of the urban landscape. The main disadvantage of this approach comes from the
difficulty of the  survey design, variables need to be chosen carefully and the process its
vulnerable to biases from the model designer.

\subsection{Pure machine learning approaches.}

\subsection{Mixed approaches.}

\section{ Explainability in deep learning.}