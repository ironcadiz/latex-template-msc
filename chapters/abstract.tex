Urban perception has been an important research subject for at least 60 years, with studies
being conducted by many different disciplines, using a variety of methodologies mainly based
on surveys over either real or simulated urban environments. The recently increased availability
of large amounts of data and highly scalable data collection methods powered by the modern web
has allowed for new techniques from other domains to be extended to the estimation of urban perception.
In particular, machine learning methods used as either stand alone models or feature extraction tools
have proven to be very effective for automatic quantification of the perception. This methods (neural networks in particular) present the disadvantage of having a black box
nature, which can make it hard for humans to understand the obtained results, therefore limiting
their application.

In this work we present a novel neural network architecture for automatic urban perception quantification.
Our best model, named AttnSegRank, can output an estimated urban perception score based on an image,
along with a set of weights (displayable as a heatmap) that reflect the importance of each part of the image
on the calculation of the score. It achieves this by including the output of a pretrained
semantic segmentator leveraged with an attention mechanism as part of the architecture. The model
we show in this work presents very similar performance with those in the previous literature but
with a much better interpretability, making it  not only a more useful model for urban perception
measuring and research, but a contribution to explainability in
the deep learning and computer vision fields that can be applied to other tasks as well.


\vfill
\noindent {\bf Keywords}:  urban perception, deep learning, explainable artificial intelligence.
