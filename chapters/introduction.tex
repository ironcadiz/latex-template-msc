\textit{Urban perception} is a feeling held by people about a location. These feelings can be and
are often related to a particular characteristic, like happiness or beauty, or also
inherently negative ones, like insecurity or fear \cite{tamara_judgments}. Understanding the
cause of these feelings is a complex task, since unique social and psychological aspects
of each individual affect how they perceive and the spaces they observe \cite{nasar_perception}.

Visual urban perception is  responsible for a large parte of the experience that people
go through while  being at or using an urban space, this not only affects how much the spaces
themselves are used \cite{khisty} but also the use of related means of transport \cite{antonakos}.
Other studies have also found correlations between urban perception, crime statistics \cite{tamara_judgments}
and wealth, and therefore used it as a proxy measure of inequality \cite{tamara_judgments,hidalgo_inequality, rossetti}.

On the other hand, being able to understand a community's need and perception of a city at scale is something
of key importance on developing cities, so that the limited  resources of local governments can be applied
more efficiently \cite{santani}.

Traditional methods for obtaining this type of data, consist of hand made polls
about specific locations making systematic evaluation of perception  an extremely costly and hard
to escalate task \cite{nasar_perception,clifton}. Other approach consist of surveys based on
computer generated images of simulated spaces, this is more scalable, but is limited to experimental
design and it doesn't apply to a real space \cite{lain_grenspace,iglesias_perception}.

Currently, thanks to the great volumes of data generated by web platforms \cite{hidalgo_inequality}
and to modern deep learning (DL) and computer vision techniques \cite{lecun_dl}, new solutions for
estimating urban perception have become feasible, and some previous studies have achieved
significant results, either by applying traditional deep learning \cite{hidalgo_placepulse}
or by combining it with other approaches \cite{rossetti, zhang_measuring}. The solutions consist
mainly of training deep convolutional neural network models (DCNN) \cite{lecun_mnist} with datasets
of urban images that have some sort of label that is used as an estimator for the perception
of that urban space. Most of the research is based on the place pulse dataset \cite{hidalgo_placepulse}, which
consists of pairs of images along with labels that indicate which of the images is more representative of a
particular attribute.

However, current deep learning methodologies, have the disadvantage of being "black boxes", in other
words, they lack a direct or systematic way to explain or interpret the obtained results. This problem
comes from the end to end nature of the neural network models and from the millions of learnable parameters
they contain. Many of the problems in which these models are used would greatly benefit of more
human understandable explanations of the results, making this a very important area of
research for the deep learning field \cite{adadi_xai}.

For the particular case of urban perception, explainability of the results is highly relevant, since
the added information is valuable for the design of public policy, for example, it could be use to
better discriminate which locations would be better recipients of an intervention, and which elements
to modify so it convenes an effective improvement of perception.

Current research in explainability is primarily moving in two directions: one is to design novel neural network architectures and training methods so
the models are more interpretable, such as the work by \citeA{dong_semantic}, the other
direction is to  create post-hoc algorithms \cite{adadi_xai} that analyze the results given by the
neural network, these algorithms sometimes use other machine learning models, including neural networks \cite{kim_ace}.

The work by \citeA{rossetti}, presents an approach to this problem for the urban perception case
by using semantic segmentation of the images \cite{segnet} as input for a discrete choice model that
estimates the perception. The approach allows for a post-hoc aggregated analysis of the results, since
the weights of the model are measure of the importance of each class of the semantic segmentation in the
calculation of the perception.

The objective of this work is to design and train a model for the urban perception problem,
that can give explainable insights on an instance level. For that it proposes a novel solution,
consisting of a neural network architecture, that is end-to-end trainable and by using semantic
segmentation \cite{pspnet} and self attention mechanisms \cite{vaswani_attention} can show
explainable insights for each of the input images.

ESTO HAY QUE ARREGLARLO AL FINAL \\
The remainder of this manuscript is organized as follows, Chapter 3 summarizes relevant previous research. In chapter 4
the problem is formally defined and the proposed model is described. Chapter 5 gives
details on model implementation and training. Finally, in chapter 6 presents the research results
and 7 the final conclusion.
