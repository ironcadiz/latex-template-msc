La percepción urbana ha sido un tema de investigación importante por al menos 60 años, con trabajos
hechos desde una variedad de disciplinas, usando distintas metodologías principalmente basadas en encuestas
sobre ambientes urbanos reales o simulados. Recientemente, el surgimiento de las tecnologías web ha aumentado
masivamente la disponibilidad de grandes volúmenes de datos y de técnicas de recolección de datos altamente escalables
permitiendo que técnicas pertenecientes a otros dominios sean utilizadas para estimar la percepción urbana. En
particular, algunos métodos de aprendizaje de máquina, usados ya sea como modelos completos o como herramientas
de extracción de características, han demostrado ser muy efectivos para la cuantificación automática de la percepción.
Estos métodos (en particular las redes neuronales) presentan la desventaja de tener una naturaleza caja negra,
que dificulta la capacidad de entender los resultados obtenidos desde el punto de vista humano, lo que
limita su aplicabilidad.

En este trabajo presentamos una nueva arquitectura de red neuronal para la cuantificación de la percepción urbana.
A partir de una imagen nuestro mejor modelo, \textit{AttnSegRank}, entrega como resultado una estimación de la percepción,
junto a un conjunto de pesos que  (visualizables como un mapa de calor) que reflejan la importancia de cada parte de
la imagen en el calculo del score. Esto se lo logra incluyendo un segmentador semántico combinado con un mecanismo
de atención como parte de la arquitectura de la red. El modelo que mostramos en este trabajo logra un rendimiento similar
a los existentes en la literatura pero con  mucho mejor interpretabilidad,  haciendolo no solo un modelo más útil para
medición e investigación de la percepción urbana, si no que una contribución a la explicabilidad en aprendizaje profundo
y visión por computador que se puede aplicar a otras tareas.

\vfill
\noindent {\bf Palabras Claves}: percepción urbana, aprendizaje profundo, inteligencia artificial explicable.
